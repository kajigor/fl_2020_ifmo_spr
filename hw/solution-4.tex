\documentclass[12pt]{article}
\usepackage[left=2cm,right=2cm,top=2cm,bottom=2cm,bindingoffset=0cm]{geometry}
\usepackage[utf8x]{inputenc}
\usepackage[english,russian]{babel}
\usepackage{cmap}
\usepackage{amssymb}
\usepackage{amsmath}
\usepackage{url}
\usepackage{pifont}
\usepackage{tikz}
\usepackage{verbatim}

\usetikzlibrary{shapes,arrows}
\usetikzlibrary{positioning,automata}
\tikzset{every state/.style={minimum size=0.2cm},
initial text={}
}


\newenvironment{myauto}[1][3]
{
  \begin{center}
    \begin{tikzpicture}[> = stealth,node distance=#1cm, on grid, very thick]
}
{
    \end{tikzpicture}
  \end{center}
}


\begin{document}
\begin{center} {\LARGE Формальные языки} \end{center}

\begin{center} \Large домашнее задание до 23:59 16.03 \end{center}
\begin{center}
  Выполнил: Кальницкий Владимир
\end{center}
\bigskip

\begin{enumerate}
  \item Доказать или опровергнуть свойство регулярных выражений:
  \[
    \forall p, q \text{ --- регулярные выражения}: (p \mid q)^* = p^*(qp^*)^*
    \]

    Автомат для первого регулярного выражения (здесь и далее рёбра помечаются не символом, а целым регулярным выражением):

    \begin{myauto}
    \node[state,initial,accepting]   (q_1) at (0,2) {$q_1$};
    \node[state,accepting] (q_2) at (2,0) {$q_2$};
    \node[state,accepting] (q_3) at (4,0) {$q_3$};

    \path[->] (q_1) edge              node [above] {$p$}    (q_2)
              (q_1) edge              node [above] {$q$}    (q_3)
              (q_2) edge              node [above] {$q$}    (q_3)
              (q_2) edge[loop below]              node [] {$p$}    ()
              (q_3) edge[bend left] node [below] {$p$}    (q_2)
              (q_3) edge[loop above]             node [] {$q$}    ()
    ;
    % \draw[->] (q_3) .. controls +(down:7mm) and +(right:7mm) .. (q_2);
  \end{myauto}

    Минимальный ДКА:

    \begin{myauto}
    \node[state,initial,accepting]   (q_1) at (0,2) {$q_1$};

    \path[->] (q_1) edge[loop above]             node [] {$p,q$}    ()
    ;
    % \draw[->] (q_3) .. controls +(down:7mm) and +(right:7mm) .. (q_2);
    %
    %

    \end{myauto}

    Автомат для второго регулярного выражения:


    \begin{myauto}
    \node[state,initial,accepting]   (q_1) at (0,2) {$q_1$};
    \node[state,accepting] (q_2) at (2,0) {$q_2$};
    \node[state,accepting] (q_3) at (4,0) {$q_3$};
    \node[state,accepting] (q_4) at (4,2) {$q_4$};

    \path[->] (q_1) edge              node [above] {$p$}    (q_2)
              (q_1) edge              node [above] {$q$}    (q_3)
              (q_2) edge              node [above] {$q$}    (q_3)
              (q_2) edge[loop below]              node [] {$p$}    ()
              (q_3) edge[bend left] node [below] {$p$}    (q_4)
              (q_3) edge[loop below]             node [] {$q$}    ()
              (q_4) edge[bend left] node [below] {$q$}    (q_3)
              (q_4) edge[loop above]             node [] {$p$}    ()
    ;
    % \draw[->] (q_3) .. controls +(down:7mm) and +(right:7mm) .. (q_2);
  \end{myauto}


    Минимальный ДКА:

    \begin{myauto}
    \node[state,initial,accepting]   (q_1) at (0,2) {$q_1$};

    \path[->] (q_1) edge[loop above]             node [] {$p,q$}    ()
    ;
    % \draw[->] (q_3) .. controls +(down:7mm) and +(right:7mm) .. (q_2);
    %
    %

    \end{myauto}

    Минимальные ДКА совпадают, поэтому регулярные выражения эквивалентны.

  \item Доказать или опровергнуть свойство регулярных выражений:
  \[
    \forall p, q \text{ --- регулярные выражения}: (p q)^* p = p (q p)^*
    \]

    Автомат для первого регулярного выражения (здесь и далее рёбра помечаются не символом, а целым регулярным выражением):

    \begin{myauto}
    \node[state,initial]   (q_1) at (0,2) {$q_1$};
    \node[state,accepting]           (q_2) at (2,0) {$q_2$};
    \node[state] (q_3) at (4,0) {$q_3$};

    \path[->] (q_1) edge              node [above] {$p$}    (q_2)
              (q_2) edge[bend left] node [above] {$q$}    (q_3)
              (q_3) edge[bend left] node [below] {$p$}    (q_2)
    ;
    % \draw[->] (q_3) .. controls +(down:7mm) and +(right:7mm) .. (q_2);
  \end{myauto}

    Минимизированный:

    \begin{myauto}
    \node[state,initial]   (q_1) at (0,2) {$q_1$};
    \node[state,accepting]           (q_2) at (2,0) {$q_2$};

    \path[->] (q_1) edge              node [above] {$p$}    (q_2)
              (q_2) edge[bend left=20]              node [below] {$q$}    (q_1)
    ;
    % \draw[->] (q_3) .. controls +(down:7mm) and +(right:7mm) .. (q_2);
  \end{myauto}

    Автомат для второго регулярного выражения:

    \begin{myauto}
    \node[state,initial]   (q_1) at (0,2) {$q_1$};
    \node[state,accepting]           (q_2) at (2,0) {$q_2$};
    \node[state]            (q_3) at (4,0) {$q_3$};
    \node[state,accepting] (q_4) at (2,2) {$q_4$};

    \path[->] (q_1) edge              node [above] {$p$}    (q_2)
              (q_2) edge[bend right] node [above] {$q$}    (q_3)
              (q_3) edge[bend left] node [below] {$p$}    (q_4)
              (q_4) edge[bend left] node [below] {$q$}    (q_3)
    ;
    % \draw[->] (q_3) .. controls +(down:7mm) and +(right:7mm) .. (q_2);
  \end{myauto}

    Минимизированный:

    \begin{myauto}
    \node[state,initial]   (q_1) at (0,2) {$q_1$};
    \node[state,accepting]           (q_2) at (2,0) {$q_2$};

    \path[->] (q_1) edge              node [above] {$p$}    (q_2)
              (q_2) edge[bend left=20]              node [below] {$q$}    (q_1)
    ;
    % \draw[->] (q_3) .. controls +(down:7mm) and +(right:7mm) .. (q_2);
  \end{myauto}

    Минимизированные ДКА совпадают, поэтому регулярные выражения эквивалентны.

  \item Доказать или опровергнуть свойство регулярных выражений:
  \[
    \forall p, q \text{ --- регулярные выражения}: (p q)^* = p^* q^*
    \]

    Опровержение: при p = 'a', q = 'b', первое выражение принимает строку abab, второе -- нет.

  \item Для регулярного выражения:
   \[ (a \mid b)^+ (aa \mid bb \mid abab \mid baba)^* (a \mid b)^+\]
  Построить эквивалентные:
  \begin{enumerate}
    \item Недетерминированный конечный автомат
    \item Недетерминированный конечный автомат без $\varepsilon$-переходов
    \item Минимальный полный детерминированный конечный автомат

      Решение:

      (a|b)+ принимает любую цепочку символов a и b ненулевой длины.

      Любой суффикс произвольной строки, которая может быть принята \[ (aa \mid bb \mid abab \mid baba)^*\], может быть принят (a|b)+.

      Поэтому \[(aa \mid bb \mid abab \mid baba)^*\] можно просто убрать из регулярного выражения:

   \[ (a \mid b)^+ (aa \mid bb \mid abab \mid baba)^* (a \mid b)^+ = (a \mid b)^+ (a \mid b)^+ \]

   Поэтому минимальный ДКА для этого регулярного выражения такой:

    \begin{myauto}
    \node[state,initial]   (q_1) at (0,2) {$q_1$};
    \node[state]           (q_2) at (2,0) {$q_2$};
    \node[state,accepting] (q_3) at (2,2) {$q_3$};

    \path[->] (q_1) edge node [above] {$a,b$}    (q_2)
              (q_2) edge node [above] {$a,b$}    (q_3)
              (q_3) edge [loop above]    node         {$a,b$} ()

    ;
    % \draw[->] (q_3) .. controls +(down:7mm) and +(right:7mm) .. (q_2);
  \end{myauto}


  \end{enumerate}

\end{enumerate}

\end{document}
