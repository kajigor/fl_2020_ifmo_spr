\documentclass[12pt]{article}
\usepackage[left=2cm,right=2cm,top=2cm,bottom=2cm,bindingoffset=0cm]{geometry}
\usepackage[utf8x]{inputenc}
\usepackage[english,russian]{babel}
\usepackage{cmap}
\usepackage{amssymb}
\usepackage{amsmath}
\usepackage{url}
\usepackage{pifont}
\usepackage{tikz}
\usepackage{verbatim}
\usepackage{amsthm}

\usetikzlibrary{shapes,arrows}
\usetikzlibrary{positioning,automata}
\tikzset{every state/.style={minimum size=0.2cm},
initial text={}
}


\newenvironment{myauto}[1][3]
{
  \begin{center}
    \begin{tikzpicture}[> = stealth,node distance=#1cm, on grid, very thick]
}
{
    \end{tikzpicture}
  \end{center}
}


\begin{document}
\begin{center} {\LARGE Формальные языки} \end{center}

\begin{center} \Large домашнее задание до 23:59 16.03 \end{center}
\bigskip

\begin{enumerate}
  \item Доказать или опровергнуть свойство регулярных выражений:
  \[
    \forall p, q \text{ --- регулярные выражения}: (p \mid q)^* = p^*(qp^*)^*
  \]

  \item Доказать или опровергнуть свойство регулярных выражений:
  \[
    \forall p, q \text{ --- регулярные выражения}: (p q)^* p = p (q p)^*
  \]

  \begin{proof}

    \[
      \forall p, q \text{ --- регулярные выражения}: (p q)^k p = p (q p)^k
    \]
    Индукция по k:
    \begin{description}
      \item База:
        \[k = 0,   p = p\]
      \item Переход:
        \[(pq)^k p = p (qp)^k \]
        \[ pq ~ (pq)^{k-1}p = pq~p(qp)^{k-1}\]
        Верно, исходя из индукционного предположения:
        \[(pq)^{k-1}p = p(qp)^{k-1}\]
        Очевидно:
        \[pq = pq\]

    \end{description}
    \[\]
  \end{proof}
  \item Доказать или опровергнуть свойство регулярных выражений:
  \[
    \forall p, q \text{ --- регулярные выражения}: (p q)^* = p^* q^*
  \]

  \begin{proof}
    Контрпример, строка разбираемая лишь одним из двух выражений:
    \[
      p
    \]

  \end{proof}

  \item Для регулярного выражения:
   \[ (a \mid b)^+ (aa \mid bb \mid abab \mid baba)^* (a \mid b)^+\]
  Построить эквивалентные:
  \begin{enumerate}
    \item Недетерминированный конечный автомат
    \begin{myauto}
      \node[state,initial]   (q_0) {$q_0$};
      \node[state]           (q_1) [right=of q_0] {$q_1$};
      \node[state]           (q_2) [above right=of q_1] {$q_2$};
      \node[state]           (q_3) [below right=of q_1] {$q_3$};
      \node[state]           (q_4) [above right=of q_2] {$q_4$};
      \node[state]           (q_5) [right=of q_2] {$q_5$};
      \node[state]           (q_6) [right=of q_3] {$q_6$};
      \node[state]           (q_7) [below right=of q_3] {$q_7$};
      \node[state]           (q_8) [right=of q_4] {$q_8$};
      \node[state]           (q_9) [right=of q_7] {$q_9$};
      \node[state]           (q_10) [right=of q_8] {$q_{10}$};
      \node[state]           (q_11) [right=of q_9] {$q_{11}$};
      \node[state, accepting] (q_12) [below right=of q_5] {$q_{12}$};


      \path[->] (q_0) edge              node [above] {$a, b$} (q_1)

                (q_1) edge              node [above] {$a$} (q_2)
                (q_1) edge              node [above] {$b$} (q_3)
                (q_1) edge [bend right=90] node [above] {$\epsilon$} (q_0)
                (q_1) edge              node [above] {$a, b$} (q_12)

                (q_2) edge              node [above] {$b$} (q_4)
                (q_2) edge              node [above] {$a$} (q_5)

                (q_3) edge              node [above] {$b$} (q_6)
                (q_3) edge              node [above] {$a$} (q_7)

                (q_4) edge              node [above] {$a$} (q_8)

                (q_5) edge              node [above] {$\epsilon$} (q_1)
                (q_5) edge              node [above] {$a, b$} (q_12)

                (q_6) edge              node [above] {$\epsilon$} (q_1)
                (q_6) edge              node [above] {$a, b$} (q_12)

                (q_7) edge              node [above] {$b$} (q_9)

                (q_8) edge              node [above] {$b$} (q_10)

                (q_9) edge              node [above] {$a$} (q_11)

                (q_10) edge [bend right=60]             node [above] {$\epsilon$} (q_1)
                (q_10) edge              node [right] {$a,b$} (q_12)

                (q_11) edge [bend left=60]             node [below] {$\epsilon$} (q_1)
                (q_11) edge              node [right] {$a,b$} (q_12)

                (q_12) edge [loop right]             node [above] {$a, b$} ()

      ;
    \end{myauto}
    \item Недетерминированный конечный автомат без $\varepsilon$-переходов
    \item Минимальный полный детерминированный конечный автомат
  \end{enumerate}

\end{enumerate}

\end{document}
