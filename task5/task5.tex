\documentclass[12pt]{article}
\usepackage[left=2cm,right=2cm,top=2cm,bottom=2cm,bindingoffset=0cm]{geometry}
\usepackage[T2A,T1]{fontenc}
\usepackage[utf8]{inputenc}
\usepackage[english, russian]{babel}

\begin{document}

\bigskip
\begin{enumerate}

\item Существует ли детерминированный конечный автомат, порождающий язык
    $\{ a^k \omega b^k | k \geq 0, \omega \in \{a, b\}^{*}, |\omega|_a \leq 3\}$.
Если да, построить; если нет — обосновать.

Утверждается, что ДКА, порождающего представленный язык L не существует, иными словами язык не является регулярным.

Допустим L является регулярным.

        %%        Тогда, по лемме о накачке для регулярных языков, $\exists$ такое n $\in$ N, что $\forall w \in L$, таких что $|w| \geq n$, существуют такие $x, y, z \in \Sigma^{*}$,
        %%
        %%что $xyz=w$, при том что $|xy| \leq n, y \neq \varepsilon$ и $\forall i \in Z_{0+}, xy^iz \in L$.
        %%
Тогда, по лемме о накачке для регулярных языков, существует такое натуральное число $n$,
что для любого слова $w \in L$, такого что $|w| \geq n$, существуют такие $x, y, z \in \Sigma^*$,
что $xyz=w$, при том что $|xy| \leq n, y \neq \varepsilon$ и $\forall i \in Z_{0+}, xy^iz \in L$.

Утверждается, что для любого предоставленного $n$ можно представить слово $\in L$, но для которого вышеупомянутые свойства не выполняются.

Подобным словом будет: $k=n, \omega = aaa$ ( $a^kaaab^k$ )

Так как $|xy| \leq n$, то $y$ в этом слове состоит из 'a', однако при $i>n$ полученная строка не будет принадлежать $L$, потому что
\begin{enumerate}
        \item $ a^m~aaa~b^k,~m > k $, 
        \item $a^k~a^{m-k}~aaa~b^k,~ \omega= a^{m-k}aaa$
        \item однако, так как $i>n$, то $m - k \geq 1 $,
        \item что противоречит правилу $|w|_a \leq 3$.
\end{enumerate}

Из этого можно заключить, что необходимое для регулярности языкa свойство не выполняется, а значит язык не является регулярным
и, следовательно, ДКА, порождающего этот язык, не существует.

\end{enumerate}

\end{document}
